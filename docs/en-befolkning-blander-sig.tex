% Options for packages loaded elsewhere
\PassOptionsToPackage{unicode}{hyperref}
\PassOptionsToPackage{hyphens}{url}
%
\documentclass[
]{book}
\usepackage{amsmath,amssymb}
\usepackage{iftex}
\ifPDFTeX
  \usepackage[T1]{fontenc}
  \usepackage[utf8]{inputenc}
  \usepackage{textcomp} % provide euro and other symbols
\else % if luatex or xetex
  \usepackage{unicode-math} % this also loads fontspec
  \defaultfontfeatures{Scale=MatchLowercase}
  \defaultfontfeatures[\rmfamily]{Ligatures=TeX,Scale=1}
\fi
\usepackage{lmodern}
\ifPDFTeX\else
  % xetex/luatex font selection
\fi
% Use upquote if available, for straight quotes in verbatim environments
\IfFileExists{upquote.sty}{\usepackage{upquote}}{}
\IfFileExists{microtype.sty}{% use microtype if available
  \usepackage[]{microtype}
  \UseMicrotypeSet[protrusion]{basicmath} % disable protrusion for tt fonts
}{}
\makeatletter
\@ifundefined{KOMAClassName}{% if non-KOMA class
  \IfFileExists{parskip.sty}{%
    \usepackage{parskip}
  }{% else
    \setlength{\parindent}{0pt}
    \setlength{\parskip}{6pt plus 2pt minus 1pt}}
}{% if KOMA class
  \KOMAoptions{parskip=half}}
\makeatother
\usepackage{xcolor}
\usepackage{longtable,booktabs,array}
\usepackage{calc} % for calculating minipage widths
% Correct order of tables after \paragraph or \subparagraph
\usepackage{etoolbox}
\makeatletter
\patchcmd\longtable{\par}{\if@noskipsec\mbox{}\fi\par}{}{}
\makeatother
% Allow footnotes in longtable head/foot
\IfFileExists{footnotehyper.sty}{\usepackage{footnotehyper}}{\usepackage{footnote}}
\makesavenoteenv{longtable}
\usepackage{graphicx}
\makeatletter
\def\maxwidth{\ifdim\Gin@nat@width>\linewidth\linewidth\else\Gin@nat@width\fi}
\def\maxheight{\ifdim\Gin@nat@height>\textheight\textheight\else\Gin@nat@height\fi}
\makeatother
% Scale images if necessary, so that they will not overflow the page
% margins by default, and it is still possible to overwrite the defaults
% using explicit options in \includegraphics[width, height, ...]{}
\setkeys{Gin}{width=\maxwidth,height=\maxheight,keepaspectratio}
% Set default figure placement to htbp
\makeatletter
\def\fps@figure{htbp}
\makeatother
\setlength{\emergencystretch}{3em} % prevent overfull lines
\providecommand{\tightlist}{%
  \setlength{\itemsep}{0pt}\setlength{\parskip}{0pt}}
\setcounter{secnumdepth}{5}
\usepackage{booktabs}
\usepackage{amsthm}
\makeatletter
\def\thm@space@setup{%
  \thm@preskip=8pt plus 2pt minus 4pt
  \thm@postskip=\thm@preskip
}
\makeatother
\ifLuaTeX
  \usepackage{selnolig}  % disable illegal ligatures
\fi
\usepackage[]{natbib}
\bibliographystyle{apalike}
\IfFileExists{bookmark.sty}{\usepackage{bookmark}}{\usepackage{hyperref}}
\IfFileExists{xurl.sty}{\usepackage{xurl}}{} % add URL line breaks if available
\urlstyle{same}
\hypersetup{
  pdftitle={En befolkning blander sig},
  pdfauthor={Christian Albrekt Larsen og Hans-Peter Y. Qvist (Red.); Med bidrag fra Jeppe Fjeldgaard Larsen, Laciné E. Diop,\ldots{} Anders? Anna?},
  hidelinks,
  pdfcreator={LaTeX via pandoc}}

\title{En befolkning blander sig}
\author{Christian Albrekt Larsen og Hans-Peter Y. Qvist (Red.) \and Med bidrag fra Jeppe Fjeldgaard Larsen, Laciné E. Diop,\ldots{} Anders? Anna?}
\date{2024-04-19}

\begin{document}
\maketitle

{
\setcounter{tocdepth}{1}
\tableofcontents
}
\hypertarget{forord}{%
\chapter*{Forord}\label{forord}}
\addcontentsline{toc}{chapter}{Forord}

Måske et forord her.

Samt en smart \emph{måde} at præsenterer \textbf{indholdsfortegnelse} og \textbf{\emph{overblik}} i dette format. Evt. korte summaries/abstracts til de enkelte kapitler?

Vi kan linke til andre \protect\hyperlink{kap1}{kapitler} og refencer i henhold til bibtex standarder \citep{xie2015}.

\hypertarget{kap1}{%
\chapter{En befolkning blander sig}\label{kap1}}

\begin{figure}
\includegraphics[width=24.89in]{images/dalle-smeltedige} \caption{Smeltedigen, tolket af en AI model}\label{fig:fig-smelte}
\end{figure}

\hypertarget{kap2}{%
\chapter{Partnerskabet og de blandede børn}\label{kap2}}

\begin{figure}
\includegraphics[width=24.89in]{images/dalle-wedding} \caption{Et blandet ægteskab, tolket af en AI model}\label{fig:fig-partner}
\end{figure}

\hypertarget{kap3}{%
\chapter{Grundskoler som mødested}\label{kap3}}

\begin{figure}
\includegraphics[width=24.89in]{images/dalle-schoolseg} \caption{Skolesegregering som det tolkes af en AI model}\label{fig:fig-schoolseg}
\end{figure}

\hypertarget{introduktion}{%
\section{Introduktion}\label{introduktion}}

Den danske grundskole er et centralt mødested mellem danskfødte, der er kendetegnet ved hvid hudfarve og kulturkristendom, og etniske minoriteter, hvoraf mange ikke deler disse kendetegn. Med andre ord, grundskolen er et rum hvor børn kan interagere med hinanden, uanset deres ligheder med andre børn og deres forældre.

Grundskolen i Danmark omfatter børn i alderen 6-16 år. Et særligt kendetegn ved den danske grundskole, sammenlignet med andre internationale skolesystemer, er fraværet af ``tracking''---elevdifferentiering--baseret på faglige evner. Det betyder, at børn ikke bliver placeret på bestemte skoler eller spor afhængigt af deres præstationer i de tidlige skoleår, som det er tilfældet i andre europæiske lande som Tyskland, Holland og England. I stedet fastlægger folkeskoleloven, at undervisningen i det danske skolesystem skal tilpasses til det pågældende klasserum gennem undervisningsdifferentiering.

I den internationale kontakt- og integrationslitteratur fremhæves skoler ofte som steder med potentiale til at nedbryde fordomme og stereotyper. Dette potentiale skyldes flere faktorer: 1) Børnene befinder sig i en kontekst, hvor der er en autoritet (læreren), der strukturerer interaktioner og opgaver. 2) Børnene forventes at have et fælles mål (læring/eksamener). 3) De arbejder sammen om dette mål i overensstemmelse med skolens didaktiske principper. 4) Børnene har samme status\footnote{Det skal selvfølgelig ikke underkendes at der er både forskning og personlige historier, der beskriver tilfælde af diskrimination og fordomme mellem lærer og minoritetselever og elever imellem \citep{andersen2019}.} i klassen (alle er elever underlagt læreren) \citep{allport1979, pettigrew2006, tropp2005}. Optimal social kontakt i denne kontekst, som påpeget af \citet{pettigrew1998}, er også betinget af, 5) at interaktionerne har venskabspotentiale. Da børn i en klasse har samme alder, og der som regel er en nogenlunde lige kønsfordeling i de fleste skoler, er der et principielt højt venskabspotentiale i de danske grundskoler \citep{mcpherson2001}.

International forskning har vist, at børn i såkaldte ``blandede skoler'' har flere venskaber eller sociale relationer på tværs af etniske gruppeskel \citep{kruse2019, leszczensky2015}. En anden forventet effekt er de såkaldte klassekammerateffekter (peer effect), som antager, at ressourcestærke elever kan være med til at hæve det faglige niveau for deres mindre ressourcestærke klassekammerater. Der pågår dog samtidigt diskussioner om, at både kontakt- og klassekammerateffekter i en metodologisk forstand er svære at isolere kausalt, da der forventeligt er grundlæggende problemer med selvselektion. For eksempel vil familier med de allerede laveste fordomme være mere tilbøjelige til at vælge den etnisk diverse distriktsskole (se f.eks. \citet{hassan2022} eller \citet{hermansen2015} for en oversigt og diskussion).

Med det danske princip om undervisningsdifferentiering i stedet for elevdifferentiering er den danske grundskole forventeligt et eksempel på optimal realisering af positive kontakt-effekter gennem sociale relationer på tværs af grænser i et barns formative år \citep{larsen2024a, larsen2016}, især i betragtning af, at børnene tilbringer 10 år sammen i alle fag. Den aktuelle udfordring er imidlertid, at Danmark har en meget liberal og generøs skolevalgspolitik, hvor omkring 75\% af omkostningerne for hvert enkelt barn er statsfinansierede, mens den resterende fjerdedel er brugerbetaling, hvilket gør privatskoler tilgængelige for en stor del af befolkningen - men samtidig utilgængelige for de laveste indkomstgrupper. Dette har skabt bekymringer for, at det socialdemokratiske princip om, at børn fra forskellige baggrunde går på samme skole, ikke længere bliver realiseret, fordi forældre frit kan vælge skoler til og fra.

Historisk set går retten til at bestemme over sit barns skolegang i Danmark, under myndighedernes tilsyn, tilbage til Friskoleloven fra 1855. I dag tillader reglerne, at selvom hver adresse er tilknyttet et skoledistrikt, hvor barnet har garanteret ret til indskrivning, er familier frie til at søge en anden folke-, privat- eller friskole, enten inden for kommunen eller i en anden kommune. Det eneste lovlige grundlag, en folkeskole kan afvise et barns optagelse på, er, hvis skolen ikke har plads, hvilket defineres som 28 børn i hver klasse i den pågældende årgang\footnote{Kommuner kan dog sænke dette maksimum, for at begrænse mulighederne for anvendelsen af skolevalg. Grundet den decentrale finansiering af folkeskolen har de enkelte folkeskoler også store individuelle omkostninger ved at flytte et barn til et specialtilbud og kan i praksis ikke udelukke børn fra skolen. I modsætning har privatskolerne mindre udgifter i forbindelse med henvisninger til specialtilbud idet de søger disse midler hos staten, hvor folkeskolerne skal finde midlerne i deres kommunalt allokerede budget.}. Til forskel kan fri- og privatskoler permanent bortvise børn eller afvise optagelse baseret på en individuel vurdering, hvilket folkeskoler også kunne før 2005. Disse strukturelle forhold har affødt en grundlæggende bekymring for, at frit skolevalg og det private skolemarked vil føre til stigende ulighed og segregation mellem skoler på grund af socioøkonomiske forskelle i, hvem der i størst omfang vælger -- eller er i stand til -- at benytte sig af muligheden for frit skolevalg.

\hypertarget{betingelser-for-muxf8der-i-grundskolen}{%
\subsection{Betingelser for møder i grundskolen}\label{betingelser-for-muxf8der-i-grundskolen}}

Den primære hindring for realiseringen af kontakt- eller klassekammerateffekter i barndommen er først og fremmest omfanget af skolesegregering, da det konkret forhindrer kontakt mellem grupper af børn, hvis de ikke møder hinanden i deres daglige liv \citep{kruse2017}.

Segregering er et udtryk for en fysisk adskillelse af personer fra forskellige klassificerede grupper\footnote{Disse grupper kan være defineret som majoritet/minoritet status, social status, køn, og alle andre former for identitets-, økonomiske, eller sociale faktorer.}. Typisk er segregering blevet drøftet i forhold til (etnisk) boligsegregering, som angiver i hvilket omfang personer med indvandrerbaggrund bor i de samme boligområder som danskfødte personer -- og omvendt. På samme måde udtrykker etnisk skolesegregering i hvilket omfang danskfødte børn kun går på skoler med andre danskfødte børn, og vice versa. En vigtig pointe her er, at segregering refererer til fordelingen eller spredningen af de to grupper, der sammenlignes. Det vil sige, at minoritetsgruppen ikke kan være segregeret uden at majoritetsgruppen også er det.

I det resterende af dette kapitel vil jeg beskrive og illustrere omfanget af segregering i det danske grundskolesystem fra 1985 til 2020. Selvom skolesegregering på mange måder er et velbeskrevet fænomen i den internationale litteratur (se \citet{larsen2024a} for et overblik), er disse beskrivelser fortsat ofte grovkornede, da de er baseret på aggregerede tabeldata, såsom survey-baserede census data. Med de danske registerdata er det muligt at lave en detaljeret og mere finkornet beskrivelse af fordelingen af børn mellem skoler, da disse data indeholder detaljeret information om hvert enkelt barn i alle skoler, inklusive privat- og friskoler, hvilket sjældent er tilgængeligt i den internationale segregeringslitteratur.

Det resterende af kapitlet er opdelt i fire sektioner. Første sektion præsenterer det danske skolelandskabs geografi og demografi. Anden sektion måler graden af segregering på både nationalt og kommunalt niveau. Dette efterfølges af en præsentation og diskussion af, hvordan skolesegregering skal ses som et produkt af boligsegregering. Fjerde sektion konkluderer og diskuterer implikationer.

\hypertarget{det-danske-skolelandskab}{%
\subsection{Det danske skolelandskab}\label{det-danske-skolelandskab}}

Når vi ser på det fysiske skolelandskab, det vil sige alle skoler og deres adresser, var det danske skolelandskab i 1985---det tidligste år, vi har data---bestående af 1327 skoler, hvoraf 246 var privat- eller friskoler\footnote{For skoler, der lukkede før institutionsregistreret blev centraliseret, er disse skoler manuelt kodet som hhv. folkeskole eller privat-/friskole.}. I 2020---det seneste år, vi har data for---var der samlet 1848 individuelle skoler\footnote{En teknisk bemærkning er, at nogle skoler er blevet lagt sammen som afdelinger under den samme hovedskole. Selvom de er fysisk adskilte skoler, er de afdelinger af den samme institution og har fælles ledelse. I en teknisk forstand er de den samme institution. I forhold til skolevalg betyder det, at forældre kun kan vælge eller være indskrevet på hovedskolen, men principielt ikke frit kan vælge mellem afdelingerne, selv hvis de ligger nær hinanden.}, hvoraf 545 var privat- eller friskoler\footnote{Denne optælling ser bort fra alle specialtilbud, efterskoler og lignende og inkluderer kun institutioner, der i institutionsregistret er klassificeret som folkeskole eller privat- og friskole. Med denne begrænsning af data var der omkring 2,8 millioner børn, der på mindst ét tidspunkt har været registreret i det danske skolesystem på tværs af alle klassetrin.}. I 2020 havde XX\% af alle børn, der startede i børnehaveklassen, enten første eller anden generation indvandrerbaggrund; heraf XX\% fra et ikke-vestligt land. Til sammenligning var tallene i 1985 hhv. XX\% og XX\%.

Fra et geografisk perspektiv er der stor variation i, hvor tæt skoler ligger på hinanden. Dette har en indirekte betydning for omfanget af frit skolevalg, da områder med få skoler inden for relativt kort afstand også har færre reelle muligheder for at vælge alternative til distriktskolen, som diskuteret ovenfor. Figur 3.1 visualiserer antallet af skoler inden for 2 km fra centroiden af bopælssognet\footnote{Centroiden, eller det geografiske midtpunkt, er punktet, der repræsenterer den gennemsnitlige position for et område, såsom et bopælssogn.}. Kortet fungerer på sin vis som en proxy for befolkningstæthed, da der naturligvis vil være flere skoler i områder med mange familier. Men grundet lave omkostninger ved etablering af private- eller friskoler vil nogle områder stadig have et relativt stort skolemarked, på trods af relativ lav befolkningstæthed\footnote{For at åbne en fri- eller privatskole er den eneste ikke-refunderbare direkte omkostning et gebyr på 20.000 kr. til ministeriet i forbindelse med anmeldelse om oprettelse af ny skole.}. Derfor er det sådan, at familier i urbane områder generelt har flere skoler, de potentielt kan vælge mellem. De fleste husstande i Danmark har to skoler inden for 2 km, men som strålende outliers er der i København og Frederiksberg henholdsvis 20 og 24 skoler inden for 2 km\footnote{Det skal bemærkes, at et skæringspunkt på 2 km for, hvornår skoler er tæt på bopælen, selvfølgelig delvist er et arbitrært skæringspunkt, da familier har forskellige behov, præferencer og muligheder for transport. Kortet ser dog grundlæggende ens ud, hvis vi hæver skæringsgrænserne, og der er blot et tilsvarende antal flere skoler i hver kategori.}.

\begin{figure}
\includegraphics[width=24.89in]{images/dalle-schoolseg} \caption{Gennemsnitligt antal skoler indenfor 2 km. af bopæls-adressen (A) og koncentration af børn med ikke-vestlig indvandrerbaggrund i skolealderen i 2020 (B)}\label{fig:fig-3-1}
\end{figure}

\hypertarget{kap4}{%
\chapter{Arbejdspladser som mødested}\label{kap4}}

\begin{figure}
\includegraphics[width=24.89in]{images/dalle-work} \caption{En multikulturel arbejdsplads, tolket af en AI model}\label{fig:fig-work}
\end{figure}

\hypertarget{kap5}{%
\chapter{Foreninger som mødested}\label{kap5}}

\begin{figure}
\includegraphics[width=24.89in]{images/dalle-civil} \caption{Minoriters deltagelse i civilsamfund, tolket af en AI model}\label{fig:fig-civil}
\end{figure}

\hypertarget{kap6}{%
\chapter{Venskaber -- det første skridt}\label{kap6}}

\begin{figure}
\includegraphics[width=24.89in]{images/dalle-friendships} \caption{Interetniske venskaber, tolket af en AI model}\label{fig:fig-friendships}
\end{figure}

\hypertarget{kap7}{%
\chapter{Integration i et kontaktperspektiv}\label{kap7}}

\begin{figure}
\includegraphics[width=24.89in]{images/dalle-integration} \caption{Social integration, tolket af en AI model}\label{fig:fig-integration}
\end{figure}

test \citep{xie2015}.

\hypertarget{litteraturliste}{%
\chapter*{Litteraturliste}\label{litteraturliste}}
\addcontentsline{toc}{chapter}{Litteraturliste}

\hypertarget{appendix-bilag}{%
\appendix}


\hypertarget{bilag1}{%
\chapter{Første bilag\ldots{}}\label{bilag1}}

This will be Appendix A.

\hypertarget{bilag2}{%
\chapter{Andet bilag\ldots{}}\label{bilag2}}

This will be Appendix B.

  \bibliography{book.bib,packages.bib}

\end{document}
